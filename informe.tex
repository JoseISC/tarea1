\documentclass[12pt]{article}
\usepackage[utf8]{inputenc}
\usepackage[spanish]{babel}
\usepackage{listings}
\usepackage{xcolor}
\usepackage{geometry}
\usepackage{enumitem}
\usepackage{parskip} 
\usepackage{microtype}
\usepackage{graphicx}
\usepackage{xcolor}

\definecolor{codebg}{RGB}{248,248,248}
\definecolor{codeframe}{RGB}{220,220,220}
\definecolor{codekeyword}{RGB}{0,102,204}
\definecolor{codecomment}{RGB}{0,153,0}
\definecolor{codestring}{RGB}{163,21,21}
\definecolor{codenumbers}{gray}{0.5}
\geometry{a4paper, margin=2.5cm}

\title{Trabajo 1\\ Paradigmas de la Programaci\'on}
\author{Rapha\"el Maufroy \& Jos\'e Salazar Cabello}
\date{Abril 2025}

\lstdefinestyle{customc}{
  language=C++,
  backgroundcolor=\color{codebg},
  frame=single,
  rulecolor=\color{codeframe},
  keywordstyle=\color{codekeyword}\bfseries,
  commentstyle=\color{codecomment}\itshape,
  stringstyle=\color{codestring},
  basicstyle=\ttfamily\footnotesize,
  numbers=left,
  numberstyle=\tiny\color{codenumbers},
  stepnumber=1,
  numbersep=10pt,
  breaklines=true,
  showstringspaces=false,
  tabsize=2,
  literate={á}{{\'a}}1 {é}{{\'e}}1 {í}{{\'i}}1 {ó}{{\'o}}1 {ú}{{\'u}}1
           {Á}{{\'A}}1 {É}{{\'E}}1 {Í}{{\'I}}1 {Ó}{{\'O}}1 {Ú}{{\'U}}1
           {ñ}{{\~n}}1 {Ñ}{{\~N}}1
}

\begin{document}

\begin{titlepage}
    \centering
    \vspace*{2cm}
    {\scshape\LARGE Universidad Andr\'es Bello \par}
    \vspace{1cm}
    {\scshape\Large Facultad de Ingenier\'ia\par}
    \vspace{1.5cm}
    {\Large Paradigmas de la Programaci\'on\par}
    \vspace{0.5cm}
    {\huge\bfseries Trabajo 1\\La taxonom\'ia de Bloom \par}
    \vspace{2cm}
    {\Large Autores: \\Rapha\"el Maufroy\\Jos\'e Salazar Cabello\par}
    \vspace{0.5cm}
    {\Large Profesor: Juan Calder\'on Maureira\par}
    \vspace{0.5cm}
    {\Large Fecha: Abril 2025\par}
    \vfill
    {\large A\~no Acad\'emico: 2025, Semestre 1\par}
\end{titlepage}

\section{Introducci\'on}
El presente trabajo tiene como objetivo desarrollar un sistema en C++ que permita a un usuario crear y gestionar pruebas escritas utilizando la Taxonom\'ia de Bloom como referencia principal.

Dentro del desarrollo del trabajo se busc\'o aplicar conceptos fundamentales de programaci\'on orientada a objetos, como lo son la herencia, el polimorfismo y el manejo de memoria din\'amica. Adem\'as, se consider\'o dise\~nar una soluci\'on modular y estructurada que permita un f\'acil mantenimiento y comprensi\'on del c\'odigo.

La funcionalidad central del sistema consiste en gestionar pruebas compuestas por distintos tipos de preguntas, tales como preguntas de Verdadero o Falso y preguntas de Selecci\'on M\'ultiple. Cada una de estas preguntas se encuentra asociada a un nivel taxon\'omico, lo que permite categorizar y evaluar las habilidades que se desean medir en el evaluado.

\section{Descripci\'on de la soluci\'on}
La soluci\'on se estructura en cuatro clases principales:
\begin{enumerate}
    \item La clase madre \textbf{Pregunta}, de la cual heredan tanto atributos como m\'etodos las clases \textbf{Verdadero\_Falso} y \textbf{Seleccion\_Mult}. Se define en \texttt{preguntas.h} y \texttt{preguntas.cpp}
    \item La clase \textbf{Prueba}, encargada de almacenar punteros de objetos de clase Pregunta. Se define en \texttt{prueba.h} y \texttt{prueba.cpp}
    \item La clase \textbf{Menu}, que se encarga de la l\'ogica general del programa. Se define en \texttt{utils.h} y \texttt{utils.cpp}
\end{enumerate}

\subsection{Clase Pregunta}
La clase base \texttt{Pregunta} define la estructura com\'un para todos los tipos de preguntas:

\begin{lstlisting}[style=customc]
class Pregunta {
  private:
    int n_pregunta;
    string enunciado;
    string niv_tax;
    float tiempo_est;
    Pregunta *siguiente;
  public:
    Pregunta();
    virtual ~Pregunta();
    // M\'etodos getters y setters
    virtual void set_correct_resp()=0;
};
\end{lstlisting}

Los atributos principales incluyen:
\begin{itemize}
    \item \texttt{n\_pregunta}: Identificador \'unico de la pregunta
    \item \texttt{enunciado}: Texto de la pregunta
    \item \texttt{niv\_tax}: Nivel taxon\'omico seg\'un Bloom
    \item \texttt{tiempo\_est}: Tiempo estimado para responder
\end{itemize}

\subsection{Clases Derivadas}

\subsubsection{Clase Seleccion\_Mult}
Esta clase implementa las preguntas de selecci\'on m\'ultiple:

\begin{lstlisting}[style=customc]
class Seleccion_Mult : public Pregunta {  
  private:
    string correct_resp;
    vector<string> dists;
  public:
    // Constructores y m\'etodos
};
\end{lstlisting}

\subsubsection{Clase Verdadero\_Falso}
Esta clase maneja las preguntas de verdadero o falso:

\begin{lstlisting}[style=customc]
class Verdadero_Falso : public Pregunta {
  private:
    bool correct_resp;
    string justificacion;
  public:
    // Constructores y m\'etodos
};
\end{lstlisting}

\subsection{Clase Prueba}
La clase \texttt{Prueba} gestiona un conjunto de preguntas:

\begin{lstlisting}[style=customc]
class Prueba {
  private:
    int tot_preguntas;
    float tot_tiempo;
    vector<Pregunta *> preguntas;
  public:
    // M\'etodos de gesti\'on de preguntas
};
\end{lstlisting}

\subsection{Clase Menu}
La clase \texttt{Menu} implementa la interfaz de usuario:

\begin{lstlisting}[style=customc]
class Menu {
  private:
    int opcion;
    vector<Prueba *> pruebas;
  public:
    void mostrar_menu();
    void ejecutar_opcion_1();
    // ... m\'as m\'etodos
};
\end{lstlisting}

\section{Funcionalidades Implementadas}
El sistema permite:
\begin{enumerate}
    \item Crear nuevas pruebas especificando cantidad de preguntas y tiempo total
    \item Insertar preguntas de dos tipos diferentes
    \item Eliminar preguntas espec\'ificas
    \item Modificar preguntas existentes
    \item Mostrar todas las preguntas de una prueba
    \item Eliminar pruebas completas
\end{enumerate}

\section{Conclusi\'on}
El desarrollo de este trabajo permiti\'o implementar de manera pr\'actica los principales conceptos de la programaci\'on orientada a objetos en C++. Se logr\'o crear un sistema modular y estructurado que gestiona eficientemente pruebas y preguntas siguiendo la taxonom\'ia de Bloom.

El uso de herencia y polimorfismo result\'o fundamental para manejar los diferentes tipos de preguntas de manera uniforme, mientras que el uso de memoria din\'amica permiti\'o una gesti\'on flexible de las pruebas y sus componentes.

La soluci\'on desarrollada demuestra la aplicabilidad de los conceptos de POO en problemas reales, resultando en un c\'odigo mantenible, extensible y bien estructurado.

\end{document} 